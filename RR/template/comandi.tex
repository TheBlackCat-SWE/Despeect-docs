%Generali
\newcommand{\ProjectName}{Despeect}
\newcommand{\GroupName}{TheBlackCat}
% \newcommand{\gruppoLink}{\href{URL}{NOME}} TODO
\newcommand{\Email}{\href{mailto:theblackcat.swe@gmail.com}{theblackcat.swe@gmail.com}}

%comando creazione quadrato
\newcommand\crule[3][black]{\textcolor{#1}{\rule{#2}{#3}}}
%esempi d'uso
%\crule[tableLineTwo]{3mm}{3mm} -> crea quadrato con colore tableLineTwo(definito in ../../template/template.tex)
%\crule[tableLineOne]{3mm}{3mm} -> crea quadrato con colore tableLineTwo(definito in ../../template/template.tex)
%\crule{1cm}{1cm}  -> crea quadrato nero di 1 cm 
%\crule[blue]{1cm}{1cm}  -> crea quadrato blu di 1 cm
%\crule[red!50!white!100]{1cm}{1cm} -> crea quadrato rosa di 1 cm

%fine prova inserimento comando



%Documenti
\newcommand{\AR}{Analisi dei Requisiti}
\newcommand{\DdP}{Definizione di Prodotto}
\newcommand{\LdP}{Lettera di Presentazione}
\newcommand{\MU}{Manuale Utente}
\newcommand{\MS}{Manuale Sviluppatore}
\newcommand{\MM}{Manuale Manutentore}
\newcommand{\NdP}{Norme di Progetto}
\newcommand{\PdP}{Piano di Progetto}
\newcommand{\PdQ}{Piano di Qualifica}
\newcommand{\SdF}{Studio di Fattibilità}
\newcommand{\ST}{Specifica Tecnica}
\newcommand{\VI}{Verbale Interno}
\newcommand{\VE}{Verbale Esterno}
\newcommand{\GL}{Glossario}

%Ultima versione dei documenti
\newcommand{\AdRv}{\textit{Analisi dei Requisiti v1.0.0}}
\newcommand{\NdPv}{\textit{Norme di Progetto v1.0.0}}
\newcommand{\PdPv}{\textit{Piano di Progetto v1.0.0}}
\newcommand{\PdQv}{\textit{Piano di Qualifica v1.0.0}}
\newcommand{\SdFv}{\textit{Studio di Fattibilità v1.0.0}}
\newcommand{\DdPv}{\textit{Definizione di Prodotto v1.0.0}}
\newcommand{\Glv}{\textit{Glossario v1.0.0}}
% se seguite da spazio è necessario inserire \ dopo

%Nome dei componenti del gruppo
\newcommand{\lallegro}{Luca Allegro}
\newcommand{\galbanel}{Giulia Albanello}
\newcommand{\sscaglio}{Stefano Scaglione}
\newcommand{\analesso}{Andrea Nalesso}
\newcommand{\ddisomma}{Davide Di~Somma}
\newcommand{\rdamiani}{Riccardo Damiani}

%Proponente, committente
\newcommand{\Proponente}{Mivoq SRL}
\newcommand{\Referente}{dott. Giulio Paci}
\newcommand{\Vardanega}{prof. Tullio Vardanega}
\newcommand{\Cardin}{prof. Riccardo Cardin}

% Revisioni
\newcommand{\RR}{Revisione dei Requisiti}
\newcommand{\RP}{Revisione di Progettazione}
\newcommand{\RQ}{Revisione di Qualifica}
\newcommand{\RA}{Revisione di Accettazione}

%Ruoli
\newcommand{\Admin}{Amministratore}
\newcommand{\Admins}{Amministratori}

\newcommand{\ProjectResp}{Responsabile di Progetto}
\newcommand{\ProjectResps}{Responsabili di Progetto}

\newcommand{\Resp}{Responsabile}
\newcommand{\Resps}{Responsabili}

\newcommand{\Verif}{Verificatore}
\newcommand{\Verifs}{Verificatori}

\newcommand{\Design}{Progettista}
\newcommand{\Designs}{Progettisti}

\newcommand{\Coder}{Programmatore}
\newcommand{\Coders}{Programmatori}

\newcommand{\An}{Analista}
\newcommand{\Ans}{Analisti}

\newcommand{\G}[1]{\textit{#1\textsubscript{G}}} %marca una parola come presente nel glossario; e.g. \G{Telegram}
\newcommand{\ora}[2]{#1:#2} %uso e.g. \ora{ore}{minuti}
\newcommand{\data}[3]{#3-#2-#1} %uso e.g. \data{giorno}{mese}{anno}
\newcommand{\file}[1]{\textsubscript{#1}}

%tabelle
\newcommand{\thcell}[1]{\normalsize\textbf{#1}} %formatta la cella dello heade delle tabelle (non diario)

% il conteggio delle tabelle parte da zero
\setcounter{table}{-1}

%aggiunge newline dopo paragraph
\makeatletter
\renewcommand\paragraph{%
   \@startsection{paragraph}{4}{0mm}%
      {-\baselineskip}%
      {.5\baselineskip}%
      {\normalfont\normalsize\bfseries}}
\makeatother

%aggiunge newline dopo subparagraph
\makeatletter
\renewcommand\subparagraph{\@startsection{subparagraph}{5}{0mm}%
      {-\baselineskip}%
      {.5\baselineskip}%
      {\normalfont\normalsize\bfseries}}
\makeatother

